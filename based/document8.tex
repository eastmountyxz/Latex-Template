\documentclass[conference]{IEEEtran}
\IEEEoverridecommandlockouts
% The preceding line is only needed to identify funding in the first footnote. If that is unneeded, please comment it out.
\usepackage{cite}
\usepackage{amsmath,amssymb,amsfonts}
\usepackage{algorithmic}
\usepackage{graphicx}
\usepackage{stfloats}
\usepackage{textcomp}
\usepackage{xcolor}
\def\BibTeX{{\rm B\kern-.05em{\sc i\kern-.025em b}\kern-.08em
    T\kern-.1667em\lower.7ex\hbox{E}\kern-.125emX}}
\begin{document}

% 指定图片在当前目录下figures目录下
\graphicspath{{figures/}}

\title{A Network Attack Method Based on Knowledge Graph\\}

\author{\IEEEauthorblockN{Eastmount}
\IEEEauthorblockA{\textit{School of Network} \\
\textit{CDSN University}\\
Guiyang, China \\
20200622@xxx.edu.cn}
}

\maketitle

\begin{abstract}
The security situation in cyberspace is becoming more and more complex, and the traceability of malicious code attacks has become an important technical challenge facing the security protection system....
\end{abstract}

\begin{IEEEkeywords}
knowledge graph; network attack; CNN
\end{IEEEkeywords}

\section{Introduction}
In recent years, there have been more and more cyber security incidents and malicious code attacks, which have brought serious harm to the country....

\section{Related Work}

\subsection{Research on Attack Tracing of Malicious Code}

Malicious code traceability refers to the discovery of the source of malicious code based on the characteristics of the target malicious code, as shown in Table I.

\begin{table}[htbp]
	\caption{The evolution and influence of PC Malware}
	\centering
	\begin{center}
		\begin{tabular}{cllp{6cm}}
			\hline
			Period & Malware & Type\\
			\hline
			1971 & Creeper & normal software \\
			1974 & Wabbit & normal software \\
			\hline
		\end{tabular}
		\label{tab1}
	\end{center}
\end{table}


The evolution of malware is divided into three stages, which are as follows:
	
\begin{itemize}
	\item The first stage is from 1971 to 1999...
	\item The second stage is from 2000 to 2008....
\end{itemize}

\subsection{Malicious Code Detection in Academia}

The process relationship between each stage is shown in Fig.1.

\begin{figure}[htbp]
	\centering
	\includegraphics[width=0.40\textwidth]{fig1.png}
	\caption{The system model of Malicious code traceability.}
	\label{fig1}
\end{figure}


\noindent
\textbf{Feature Extraction.} Feature extraction is the basis of the traceability analysis process...

\noindent
\textbf{Feature Preprocessing.} If the ratio value is larger, the proof is more similar, as shown in the formula (1).

\begin{equation}
J(A,B) = \frac{|A \bigcap B|}{|A \bigcup B|} \label{eq}
\end{equation}

The feature extraction process will encounter unrepresentative and non-quantifiable original features.

\section{System Model}
To make up for the shortcomings of traditional malicious code attack source tracing based on single organization and attack chain...

\subsection{Research Framework}\label{RF}
The purpose of this paper is to trace the hacker organization and the author behind the malicious code..

\subsection{Feature Extraction}

\section{Experiments}

\subsection{Datasets and evaluation indicators}

\subsection{Attack Traceability Simulation Experiment}

\section{Results}



\section*{Acknowledgment}


\begin{thebibliography}{3}
\bibitem{b1} Jiang JG, Wang JZ and Kong B, ``A summary of research on traceability of network attack source'', Journal of Information Security, vol.3, no.1, pp.111--131, 2018.
\bibitem{b2} Liu J, Su PR and Yang M, ``Overview of Software and Network Security Research'', Journal of Software, vol.29, no.1, pp.42--68, 2018.
\bibitem{b3} He R, ChenZG and Pu S, ``Research on Multi-Source Network Attack Tracing and Tracing Technology'', Communication Technology, vol.46, no.12, pp.77--81, 2013.

\end{thebibliography}


\end{document}
